
%% Role of alignments & window %%


%% Topic vs function vectors %% 



%% Usefulness of joint multilingual embeddings and evaluation %% 

% Given only sentence alignments, 



Distributional semantics is a fast developing field that concerns the establishment of a semantic vector space were words have a geometrical interpretation. We investigate how to use data in multiple languages to create a single multilingual semantic space. The word representations in this vector space may have a theoretical interest on their own right, but can also be used for tasks related to translation, such as cross-lingual information retrieval and machine translation. 

We introduce an approach to induce such cross-lingual word embeddings using sentence-aligned data, based on sentence co-occurrence. We show how this is fundamentally different from approaches that focus on local co-occurrence in an n-gram context. Without word alignments, local co-occurrence cannot be used in cross-lingual induction of word embeddings. Global (sentence-based) co-occurrence yields word embeddings that are less informative of the function of a word in a sentence, but can be quite useful in Information Retrieval (IR) tasks. We use such a task to evaluate the quality of our embeddings, namely cross-lingual document classification.

This document is structured as follows. In section~\ref{s:relatedWork}, we discuss several  approaches to multilingual distributional semantics. We introduce two existing monolingual models to obtain sentence embeddings in section~\ref{s:sentenceEmbeddings}, and extend them for the multilingual case. We explain how word embeddings can be obtained in section~\ref{s:wordEmbeddings}. The tasks we use for evaluation of the resulting embeddings are described in section~\ref{s:evaluation}, followed by the experiments we conducted and emperical results in section~\ref{s:experiments}. In section~\ref{s:discussion}, we set forth some considerations and ideas for future investigation of this topic. We conclude with some final remarks in section~\ref{s:conclusion}.

