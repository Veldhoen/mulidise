%Reproducing existing work with more languages
%New approach: based on par2vec (2 flavors)


It is not trivial to measure the quality of the multilingual word embeddings. The semantic space should be reliable for each language in isolation, and consistent across languages. 
Even the former is not easy to assess. In \cite{mikolov2013efficient}, an analogy task is introduced to this aim, which we apply to our English word embeddings.

The latter is evaluated on a real-world task of cross-lingual document classification. The models we use rely on bag-of-word representations of sentences, as explained in section~\ref{s:wordEmbeddings}. Therefore, we do not expect a fine-grained semantic analysis of sentences and words but rather capture something like `topicality'. It thus make sense to apply a document classification task, following the evaluation strategy of  \cite{klementiev2012inducing,hermann2014multilingual}.


\subsection{Word analogy task}
To evaluate word embeddings monolingually, we use the word analogy task described in \cite{mikolov2013efficient}.
The task is based on the intuition that a word like \emph{big} is similar to \emph{biggest} in the same sense that \emph{small} is to \emph{smallest}.
The goal of the task is to answer questions that complete such a relationship, such as ``What word is to \emph{small} what \emph{biggest} is to \emph{big}?''
The answer can be found by adding the difference of $[\![\mathit{biggest}]\!]$ and $[\![\mathit{big}]\!]$ to $[\![\mathit{small}]\!]$, and searching the vocabulary to find the nearest embedding.
In practice, this is done by computing $X = [\![\mathit{biggest}]\!] - [\![\mathit{big}]\!]+ [\![\mathit{small}]\!]$ and comparing all word embeddings by cosine distance.
This finds the embedding that is nearest to \emph{biggest} and \emph{small}, but furthest from \emph{big} \cite{Levy2014}.
Accuracy scores are reported for only the questions that could be answered with the vocabulary.

\subsection{Document classification - RCV}
In \cite{klementiev2012inducing} a cross-lingual document classification task is introduced. The task, that is also used in \cite{hermann2013multilingual}, is based on Reuters corpora, which has topic-annotated documents (news stories). The evaluation data is available for English and German documents that belong to a single topic, and thus the gold standard can be represented by a one-hot vector.

A vector representation is obtained for each document in the dataset. In \cite{klementiev2012inducing}, the document vector is the average of the representations of its \emph{tokens}, weighted by $idf$ score. In \cite{hermann2013multilingual}, the document vector is the average of the representations of its \emph{sentences}. We use both approaches, depending on the experimental settings.

As a classifier, we use the implementation of an averaged perceptron algorithm from \cite{klementiev2012inducing}. It is trained to predict classes (topics) from document representations. In the cross-lingual setting, the perceptron is trained for document classification in one language, and tested on data in another resulting in a classification accuracy score. If the semantic space is coherent between languages, performance should not diverge much between monolingual and cross-lingual document classification. 
%Alternatively, \cite{klementiev2012inducing} compares to wordcount features. Translations are done via glossing or machine translation. It is also common to compare to a MT baseline. 

The topics in the RCV evaluation sets belong to four topics: Corporate/Industrial, Economics, Government/Social, and Markets.
For both languages, the documents are split into train sets with 100, 200, 500, 1000, 5000 and 100000 documents, and a test set of around 5000 documents.
As a baseline, we compute chance accuracy for the majority class estimate. For both languages, the majority class was Markets, with around 46.8\% of the documents.

\subsection{TED document classification}
The WIT TED corpus \cite{cettolo2012} contains short documents with transcriptions and translations of TED talks, with topic annotations. The original distribution was aimed at machine translation, but \cite{hermann2014multilingual} propose it for a multilingual document classification task. The major advantage of this task over the previous one, is the availability of documents in many languages. It has documents in English sentence-aligned with other languages, six of which are also in the Europarl data we use for obtaining our embeddings: Spanish, French, German, Italian, Dutch, and Portuguese.

There are fifteen classification labels, i.e. topics, in this set.
%The classification labels in this set are technology, culture, science, global issues, design, business, entertainment, arts, politics, education, art, health, creativity, economics, and biology. 
Note that contrary to the previous task, a document can have more than one topic annotation. A binary classifier is thus trained for each topic, using the same system as before. Performance is reported both as F1 score. As the chance accuracy for majority class is quite high, since there are only few positive examples per class, F1 is more informative for comparing performance. 

The majority class estimate is not usable as a baseline for F1 performance: as the majority of the documents are labeled negative, precision would be zero and thus F1 too (actually, undefined). As an alternative baseline, we compare to a stochastic classifier that predicts `true' with probability $P=pos/total$. The expected number of True Positives is thus $P*pos=P^2*|X|$, the expected False Positives and False Negatives are both $P*(1-P)*|X|$. We can now compute expected F1:
\begin{align*}
F1	&=\frac{2*TP}{2*TP+FN+FP}\\
			&=\frac{2*P^2*|X|}{2*P^2*|X|+2((1-P)*P*|X|)}\\
			&=P
\end{align*} Therefore, we use the ratio of positive examples as a baseline for the performance on TED data.






%In \cite{mikolov2013exploiting}, the evaluation is performed on a test set of gold-standard word translations, again from Google Translate. The word representation in the source language is transformed using $W$, and a ranked list of the nearest words in the target language is the output. The precision at ranks 1 and 5 is reported. 
% Gaan we dat nou ook nog doen of niet?

%For visualizing the vector space, we will project a selection of words onto a plane and highlight semantic relationships.
%We will also visualize rare words and words with high variability across languages.
